 \documentclass[a4paper]{article}
% \usepackage[margin=1.25in]{geometry}
\usepackage[inner=2.0cm,outer=2.0cm,top=2.5cm,bottom=2.5cm]{geometry}
% \usepackage{ctex}
\usepackage{color}
\usepackage{graphicx}
\usepackage{amssymb}
\usepackage{amsmath}
\usepackage{amsthm}
\usepackage{bm}
\usepackage{hyperref}
\usepackage{multirow}
\usepackage{enumerate}

\newcommand{\homework}[5]{
    \pagestyle{myheadings}
    \thispagestyle{plain}
    \newpage
    \setcounter{page}{1}
    \noindent
    \begin{center}
    \framebox{
        \vbox{\vspace{2mm}
        \hbox to 6.28in { {\bf Optimization Methods \hfill #2} }
        \vspace{6mm}
        \hbox to 6.28in { {\Large \hfill #1 \hfill} }
        \vspace{6mm}
        \hbox to 6.28in { {\it Instructor: {\rm #3} \hfill Name: {\rm #4}, StudentId: {\rm #5}}}
        \vspace{2mm}}
    }
    \end{center}
    % \markboth{#4 -- #1}{#4 -- #1}
    \vspace*{4mm}
}

\newenvironment{solution}
{\color{blue} \paragraph{Solution.}}
{\newline \qed}

\begin{document}
%==========================Put your name and id here==========================
\homework{Homework 1}{Fall 2019}{Lijun Zhang}{Weikang Li}{181220031}

\paragraph{Notice}
\begin{itemize}
    \item The submission email is: \textbf{njuoptfall2019@163.com}.
    \item Please use the provided \LaTeX{} file as a template. If you are not familiar with \LaTeX{}, you can also use Word to generate a \textbf{PDF} file.
\end{itemize}



\paragraph{Problem 1: Norms}
~

A function $f:\mathbb{R}^n \rightarrow \mathbb{R}$ with $\text{dom}f=\mathbb{R}^n$ is called a \textit{norm} if
\begin{itemize}
    \item $f$ is nonnegative: $f(x) \geq 0$ for all $x \in \mathbb{R}^n$
    \item $f$ is definite: $f(x) = 0$ only if $x=0$
    \item $f$ is homogeneous: $f(tx)=|t|f(x)$, for all $x \in \mathbb{R}^n$ and $t \in \mathbb{R}$
    \item $f$ satisfies the triangle inequality: $f(x+y) \leq f(x) + f(y)$, for all $x,y \in  \mathbb{R}^n$
\end{itemize}
We use the notation $f(x) = \|x\|$. Let $\| \cdot \|$ be a norm on $\mathbb{R}^{n}$.
The associated dual norm, denoted $\| \cdot \|_*$, is defined as 
$$\|z\|_* = \sup \{z^Tx | \|x\| \leq 1\}$$

\noindent
\begin{enumerate}[a)]
    \item Prove that $\|\cdot\|_*$ is a valid norm.
    \item Prove that the dual of the Euclidean norm ($\ell_2$-norm) is the Euclidean norm, \emph{i.e.}, prove that
$$\|z\|_{2*} = \sup \{z^Tx | \|x\|_2 \leq 1\} = \|z\|_2$$.\\ (\textit{Hint:} Use Cauchy–Schwarz inequality.)
\end{enumerate}
\begin{solution}
	a). To prove a valid norm, we need to validate the character according to the definition.\\
	\begin{itemize}
    \item $f$ is nonnegative: Without loss of generality, we let $x=0$, it is obvious that $\|z\|_* = \sup \{z^Tx | \|x\| \leq 1\} \geq 0$ for all $z \in \mathbb{R}^n$
    \item $f$ is definite: $f(z) = 0$ means $\sup \{z^Tx | \|x\| \leq 1\} = 0$, so $z^Tx \leq 0$, it is easy to see that this will hold only if $x=0$ 
    \item $f$ is homogeneous: From above we have $f$ is nonnegative, so:
	\begin{equation}\nonumber   
     \ f(tz) = \|tz\|_* = \sup \{(tz)^Tx | \|x\| \leq 1\} = |t|\sup \{z^Tx | \|x\| \leq 1\} = |t| \|z\|_* = |t|f(z)
	\end{equation}    
    \item $f$ satisfies the triangle inequality: 
	\begin{equation}\nonumber
	\begin{aligned}
	\ f(y+z) &= \sup \{(y+z)^Tx | \|x\| \leq 1\} \\&= \sup \{y^Tx + z^Tx | \|x\| \leq 1\} \\&\leq \sup \{y^Tx | \|x\| \leq 1\} + \sup \{z^Tx | \|x\| \leq 1\} \\&= f(y) + f(z)
	\end{aligned}
	\end{equation}	    
, for all $y,z \in  \mathbb{R}^n$
\end{itemize}	 
	b). To prove $\|z\|_{2*} = \sup \{z^Tx | \|x\|_2 \leq 1\} = \sup \{|z^Tx| | \|x\|_2 \leq 1\} =\|z\|_2:$\\According to the Cauchy-Schwarz inequality, we have: $$|z^Tx| \leq \|z\|_2\|x\|_2 \leq \|z\|_2.$$So we just need to prove that the maximum value of $|z^Tx|$ is $\|z\|_2$. Without loss of generality, we let $x=kz$:\\if $k=0$, trivial.\\if $k \not=0$, it is obvious that $\|z\|_2\|x\|_2 = \|z\|_2$ only if $\|x\|_2 = \|kz\|_2 = k\|z\|_2 = 1$, so $k = \frac{1}{\|z\|_2}$.
\end{solution}

\paragraph{Problem 2: Affine and Convex Sets}
~

Affine sets $C_{a}$ and convex $C_{c}$ sets are the sets satisfying the constraints below:
\begin{equation}
\begin{aligned}
    \theta x_1 + (1-\theta)x_2 \in C_{a}\\
    \text{s.t. } x_1, x_2 \in C_{ a}\\
\end{aligned}
\end{equation}
\begin{equation}
\begin{aligned}
    \theta x_1 + (1-\theta)x_2 \in C_{c}\\
    \text{s.t. } x_1, x_2 \in C_{c}, 0 \leq \theta \leq 1\\
\end{aligned}
\end{equation}

\noindent
\begin{enumerate}[a)]
    % \item Can we hold that $D$ is convex if $D$ is affine?
    \item Is the set $\{  \alpha \in \mathbb{R}^k | p(0)=1, |p(t)|\leq1 \text{ for } \alpha \leq t \leq \beta\}$, where $p(t) = \alpha_1+\alpha_2t+\cdots+\alpha_kt^{k-1}$, affine?
    \item Determine if each set below is convex.
        \begin{enumerate}[1)]
            \item $\{  (x,y)\in \mathbf{R}^2_{++} | x/y \leq 1 \}$.
            \item $\{  (x,y)\in \mathbf{R}^2_{++} | x/y \geq 1 \}$.
            \item $\{  (x,y)\in \mathbf{R}^2_{+} | xy \leq 1 \}$.
            \item $\{  (x,y)\in \mathbf{R}^2_{+} | xy \geq 1 \}$.
            \item $\{  (x,y)\in \mathbf{R}^2 | y = \text{tanh}(x) = \frac{e^{x}-e^{-x}}{e^{x}+e^{-x}}\}$.
        \end{enumerate}
\end{enumerate}

\begin{solution}
    a). Not affine: We suppose that $\alpha_1,\alpha_2$ in the set, without loss of generality we set $\alpha_{1,1}, \alpha_{2,1} = 1,$then we have :
    $$|\alpha_{1,1}+\alpha_{1,2}t+\cdots+\alpha_{1,k}t^{k-1}|\leq 1$$
    $$|\alpha_{2,1}+\alpha_{2,2}t+\cdots+\alpha_{2,k}t^{k-1}|\leq 1$$
    for $\theta \in \mathbb{R}:$ $$\theta \alpha_1+(1-\theta)\alpha_2=\theta(\alpha_{1,1},\alpha_{1,2},\cdots,\alpha_{1,k})+(1-\theta)(\alpha_{2 ,1},\alpha_{2,2},\cdots,\alpha_{2,k})$$ in this case, $$|p(t)|=|\theta |\alpha_{1,1}+\alpha_{1,2}t+\cdots+\alpha_{1,k}t^{k-1}| + (1-\theta)|\alpha_{2,1}+\alpha_{2,2}t+\cdots+\alpha_{2,k}t^{k-1}||$$ let $\theta=2,\, |\alpha_{1,1}+\alpha_{1,2}t+\cdots+\alpha_{1,k}t^{k-1}|=1,\,|\alpha_{2,1}+\alpha_{2,2}t+\cdots+\alpha_{2,k}t^{k-1}|=-1$, it is obvious that $|p(t)|>1$,\\so $\theta \alpha_1+(1-\theta)\alpha_2$ is not in the set, which means the set is not convex. \\
    \\b). 1) Convex;\\2) Convex;\\3) Not convex:\quad Consider a combination $z$ of two points $x (\frac{1}{2},2)$ and $y (2,\frac{1}{2})$ in the set, and let $\theta=\frac{1}{2}$. Therefore, from $z=\theta x + (1-\theta)y$ we have $z(\frac{5}{4},\frac{5}{4})$. It is obvious that $z \not\in \{  (x,y)\in \mathbf{R}^2_{+} | xy \leq 1 \}$\\4) Convex;\\5) Not convex:\quad Consider a combination $z$ of two points $x (0,0)$ and $y (-\infty,-1)$ in the set. Therefore, from $z=\theta x + (1-\theta)y$. It is obvious that for any $0 \leq \theta \leq 1$, $z \not\in \{  (x,y)\in \mathbf{R}^2 | y = \text{tanh}(x) = \frac{e^{x}-e^{-x}}{e^{x}+e^{-x}}\}$.
\end{solution}

\paragraph{Problem 3: Examples}
~
\begin{enumerate}[a)]
\item Let $C \subseteq \mathbb{R}^n$ be the solution set of a quadratic inequality,
    \begin{equation}
        C = \{ x\in \mathbb{R}^n | x^\top Ax+b^\top x+c \leq 0\}\,,
    \end{equation}
    with $A \in \mathbb{S}^n, b \in \mathbb{R}^n, \text{and } c \in \mathbb{R}$.

    \noindent
    \begin{enumerate}[1)]
        \item Show that $C$ is convex if $A \succeq 0$.
        \item Is the following statement true? The intersection of $C$ and the hyperplane defined by $g^\top x+h=0$ is convex if $A+\lambda gg^\top \succeq 0$ for some $\lambda \in \mathbb{R}$.
    \end{enumerate}
\item The polar of $C\subseteq \mathbb{R}^n$ is defined as the set
$$C^\circ = \{ y\in \mathbb{R}^n | y^\top x\leq 1 \text{ for all }x \in C\}$$\,
    \noindent
    \begin{enumerate}[1)]
        \item Show that $C^\circ$ is convex.
        \item What is a polar of a polyhedra?
        \item What is the polar of the unit ball for a norm $||\cdot||$?
        \item Show that if $C$ is closed and convex, with $0 \in C$, then $(C^\circ)^\circ = C$
    \end{enumerate}
\end{enumerate}


\begin{solution}
    a) 1) To prove the set $C$ is convex, an approach is to prove the intersection of $C$ and any lines is convex. Let the line is defined as the set $\{x_0 + tv\,|\,t \in \mathbb{R}\},\,v,x_0 \in \mathbb{R}^n$.\\Therefore, we have:
    $$(x_0+tv)^TA(x_0+tv)+b^T(x_0+tv)+c \leq 0$$
let: $$\alpha=v^TAv,\quad\beta=b^Tv+2x_0^TAv,\quad\gamma=c+b^Tx_0+x_0^TAx_0$$
the intersection is: $$\{x_0+tv\,|\,\alpha t^2+\beta t+\gamma \leq 0\}$$ 
If the line intersects the set, the above inequality has a solution. Let's consider only the case where t is solvable.\\
if $A \succeq 0$, so $\alpha=v^Tat \geq 0$\,:\\
when $\alpha = 0$:
\begin{itemize}
    \item $\beta = 0,\,\gamma \leq 0$ : $t \in \mathbb{R}$
    \item $\beta \geq 0$ :  $t \leq \frac{-\gamma}{\beta}$
    \item $\beta \leq 0$ :  $t \geq	 \frac{-\gamma}{\beta}$
\end{itemize}
when $\alpha>0$:  $t_1 \leq t  \leq t_2$\,($t_1,t_2$ is two roots of the parabola)\\
Therefore, the intersection of $C$ and any lines defined above is convex, so $C$ is convex.\\
2) Let the set of hyperplanes $H=\{x\,|\,g^Tx+h=0\}$, we define $\delta=g^Tv,\,\epsilon=g^Tx_0+h$. Assuming that $x_0 \in H$, which means $\epsilon=g^Tx_0+h=0$.\\ So, the intersection of $C\cap H$ and lines defined above is: $$\{x_0+tv\,|\,\alpha t^2+\beta t+\gamma \leq 0,\,\delta t = 0\}$$ If $\delta\not = 0$, so $t=0$, the intersection is $\{x\}$;\\If $\delta = 0$, the intersection is $\{x_0+tv\,|\,\alpha t^2+\beta t+\gamma \leq 0\}$, from above we have: this is convex if $\alpha \geq 0 \Rightarrow v^TAv \geq 0$. This will hold if $A+\lambda gg^\top \succeq 0$ for some $\lambda \in \mathbb{R}$, which means: $$v^TAv=v^T(A+\lambda gg^\top)v \geq 0$$ 
\\b) 1) According to the definition, it is obvious that the polar is the intersection of halfspaces\,$\{y\,|\,y^Tx \leq 1\}$, so it is convex.\\
2) According to the definition of the polar and polyhedra, it is easy to see that the polar of a polyhedra is still a polyhedra.\\
3) According to the definition of the unit ball and the polar of a set:
$$\mathcal{B} = \{x \in \mathbb{R}^n\,|\,\|x\|\leq 1 \},$$
$$\therefore\mathcal{B}^\circ = \{ y\in \mathbb{R}^n | y^\top x\leq 1 \text{ for all }x \in \mathcal{B}\}=\{ y\in \mathbb{R}^n |\sup \{y^Tx | \|x\| \leq 1\}\leq 1\}=\{  y\in \mathbb{R}^n | \|y\|_* \leq 1\}$$
4) Assume that $x \in C$ and $y \in C^\circ$, so $y^Tx \leq 1$, also we have $x^Ty \leq 1$ for all $y \in C^\circ$, which means $x \in (C^\circ)^\circ$, so $C \subseteq (C^\circ)^\circ$.\\
Assume that $x \in (C^\circ)^\circ$ and $x \not\in C$, According to the Separation Theorem of Hyperplane, there must be a seperating hyperplane for C and $\{x\}$, which means for $z \in C,\,a^Tz \leq b$; $a^Tx > b$; because $0 \in C$, we have $b \geq 0$.\\
Without loss of generality, we let  $z \in C,\,a^Tz \leq 1$; $a^Tx > 1$. Therefore, $a \in C^\circ$. From the assumption above, we have $x \in (C^\circ)^\circ$, which means $x^Ta \leq 1\ i.e.\ a^Tx \leq 1$, which is contradicted with hypothesis.\\ Therefore,  $(C^\circ)^\circ = C$
\end{solution}

\paragraph{Problem 4: Operations That Preserve Convexity}
~

Suppose $\phi : \mathbb{R}^n \rightarrow \mathbb{R}^m \text{ and } \psi : \mathbb{R}^m \rightarrow \mathbb{R}^p$ are the linear-fractional functions
\begin{equation}
\phi(x) = \frac{Ax+b}{c^\top x + d}, \psi(y) = \frac{Ey+f}{g^\top y + h},
\end{equation}
with domains \textbf{dom }$\phi = \{ x | c^\top x + d > 0 \}$, \textbf{dom }$\psi = \{ y | g^\top y + h > 0 \}$. We associate with $\phi$ and $\psi$ the matrices
\begin{equation}
  \left[\begin{matrix}
   A & b \\
   c^\top & d
  \end{matrix}\right]\,,
  \left[\begin{matrix}
   E & f \\
   g^\top & h
  \end{matrix}\right]\,,
\end{equation}
respectively.

Now, consider the composition $\Gamma$ of $\phi$ and $\psi$, \emph{i.e.}, $\Gamma(x)=\psi(\phi(x))$, with domain
\begin{equation}
\textbf{dom} \Gamma = \{ x \in \textbf{dom } \phi | \phi(x) \in \textbf{dom } \psi\} \,.
\end{equation}

Show that $\Gamma$ is linear-fractional, and that the matrix associate with it is the product
\begin{equation}
  \left[\begin{matrix}
   E & f \\
   g^\top & h
  \end{matrix}\right]
  \left[\begin{matrix}
   A & b \\
   c^\top & d
  \end{matrix}\right]\,.
\end{equation}


\begin{solution}
    According to the definition: $\Gamma(x)=\psi(\phi(x))$, therefore we have, for $x \in \textbf{dom } \Gamma$,
\begin{equation}\nonumber
\begin{aligned}
\Gamma(x) &= \frac{E((Ax+b)/c^\top x + d)+f}{\,g^\top(Ax+b)/(c^\top x + d) + h}\\&= \frac{EAx + Eb + fc^\top x + fd}{\,g^\top Ax + g^\top b + hc^\top x + hd}\\&= \frac{(EA + fc^\top)x + (Eb + fd)}{(g^\top A + hc^\top)x + (g^\top b + hd)}
\end{aligned}
\end{equation}
As can be seen from the form of the upper form, $\Gamma$ is linear-fractional function, and associated with the product matrix:
\begin{equation}\nonumber
  \left[\begin{matrix}
   E & f \\
   g^\top & h
  \end{matrix}\right]
  \left[\begin{matrix}
   A & b \\
   c^\top & d
  \end{matrix}\right]\,.
\end{equation}
\end{solution}

\paragraph{Problem 5: Generalized Inequalities}
~

Let $K^*$ be the dual cone of a convex cone $K$. Prove the following
\begin{enumerate}[1)]
    \item $K^*$ is indeed a convex cone.
    \item $K_1 \subseteq K_2$ implies $K_2^* \subseteq K_1^*$.
\end{enumerate}

\begin{solution}
    a) According to the definition of $K^*$: $\{  y\,|\,x^Ty \geq 0 ,\,\forall x \in K\}$ we found that $K^*$ is the intersection of a set of homogeneous halfspaces (all halfspaces are convex) . Therefore, $K^*$ is indeed a convex cone. \\
	\\b) According to the definition : $y \in K_2^*$ infers $x^Ty \geq 0$ for all $x \in K_2$. Meanwhile, $K_2 \supseteq K_1$, which infers $x^Ty \geq 0$ for all $x \in K_1$. Therefore, $K_2^* \subseteq K_1^*$. \\
\end{solution}



\end{document}